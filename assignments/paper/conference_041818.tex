\documentclass[conference]{IEEEtran}
\IEEEoverridecommandlockouts
% The preceding line is only needed to identify funding in the first footnote. If that is unneeded, please comment it out.
\usepackage{cite}
\usepackage{amsmath,amssymb,amsfonts}
\usepackage{algorithmic}
\usepackage{graphicx}
\usepackage{textcomp}
\usepackage{xcolor}
\def\BibTeX{{\rm B\kern-.05em{\sc i\kern-.025em b}\kern-.08em
    T\kern-.1667em\lower.7ex\hbox{E}\kern-.125emX}}
\begin{document}

\title{A new algorithm for the Maximum Independent Set Problem (MIS)}

\author{\IEEEauthorblockN{ Lachlan Ashcroft }
\IEEEauthorblockA{\textit{Griffith University} \\
\textit{}\\
Gold Coast, Australia \\
lochieashcroft@griffithuni.edu.au}
}

\maketitle

\begin{abstract}
This paper contains a brand new algorithm for identifying maximum independent sets in a graph as well as its complement, 
the algorithm is called Not Interesting Because Bad Algorithm (NIBBA). A literature review is also included to give 
context on what the current research is in this domain. 
\end{abstract}

\begin{IEEEkeywords}
independent, set, MIS, algorithm
\end{IEEEkeywords}

\section{Introduction}
Given an unweighted, undirected graph $G = (V, E)$ where $V$ is set of vertices and $E$ a set of edges, an 
independent set contains vertices in which there is no two adjacent vertices. A maximal independent set is a set in 
which the inclusion of any additional vertices would introduce a connecting edge into the set. The maximum independent 
set (MIS) of a graph is the largest maximal independent set. The maximum size of the MIS is $N$ where $N$ is the number
of vertices however this is only if $G$ is a completely disconnected graph. The MIS is a NP hard optimization problem [1].

MIS is closely related to the maximum clique problem, by definition a clique in $Gc$ (complement of $G$) is an 
independent set in G, by this logic the maximum independent set can be found by solving the maximum clique problem in 
the complement of in the input graph. 

The maximum independent set of a graph has various real world applications and because of this an efficient algorithm is 
highly desired. This paper is structured as follows, section 2 contains the literature review, section 3 the XYZ algorithm, section 4 
the results and lastly the conclusion is section 5.

Tests will be ran against a subset of the DIMACS[2] in order to benchmark XYZ algorithm.

\subsection{Terminology}

The degree of a vertex $v$ shown as $d(v)$ is the number of edges (connections) of vertex v

Adjacency matrix is a matrix of dimensions $(N, N)$ where $N$ is the number of vertices, a value of 1 indicates two
vertices are adjacent.

\section{Literature Review}
A literature review was performed in order to understand the current state of the research regarding the MIS and also
for inspiration for the development of a new algorithm.


, various papers have been citied. The earliest paper is from the year XXXX
blah blah blah

As mentioned in the introduction the MIS and the maximum clique problem are complementary, as a result many approaches
simply solve for the maximum clique in the complement of the input graph.


\subsection{Maintaining the Integrity of the Specifications}

\section{Algorithm}

The algorithm takes inspiration from ......

\subsection{Design}

Just some ideas which may be explored

\subsection{Parallel Greedy}

Sort the vertices by their degree in ascending order, randomly pick a starting vertex from the first 10\% of nodes. 
Randomly pick another vertex to add to the set while maintaining the independence with a bias for a vertex with a low 
degree. This is performed on multiple threads.

\subsection{Random shrink and grow}

Randomly choose $N$ vertices to build a set, if the set is independent randomly add 1 or more vertices to the set.
If the set is not independent randomly remove vertices until it is.

\subsection{Pseudocode}

\section{Results}

\begin{thebibliography}{00}
\bibitem{b1}  Garey. M. R, Johnson. D. S: Computers and Intractability: A Guide to the theory NP – completeness. San Francisco: Freeman (1979).  
\bibitem{b2} DIMACS clique benchmarks. Benchmark instances made available by electronic transfer at dimacs.rutgers.edu, Rutgers Univ., Piscataway. NJ. (1993). 
\end{thebibliography}
\end{document}
